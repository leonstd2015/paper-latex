

\documentclass[preprint,12pt]{elsarticle}



\usepackage{amssymb}


\journal{Nuclear Physics B}

\begin{document}


\begin{frontmatter}

%% Title, authors and addresses

%% use the tnoteref command within \title for footnotes;
%% use the tnotetext command for theassociated footnote;
%% use the fnref command within \author or \address for footnotes;
%% use the fntext command for theassociated footnote;
%% use the corref command within \author for corresponding author footnotes;
%% use the cortext command for theassociated footnote;
%% use the ead command for the email address,
%% and the form \ead[url] for the home page:
%% \title{Title\tnoteref{label1}}
%% \tnotetext[label1]{}
%% \author{Name\corref{cor1}\fnref{label2}}
%% \ead{email address}
%% \ead[url]{home page}
%% \fntext[label2]{}
%% \cortext[cor1]{}
%% \address{Address\fnref{label3}}
%% \fntext[label3]{}

\title{A novel 1000MW double reheat ultra supercritical system with turbine–extraction-heated air preheaters and low temperature economizers}

%% use optional labels to link authors explicitly to addresses:
%% \author[label1,label2]{}
%% \address[label1]{}
%% \address[label2]{}

\author[rvt]{Lei Zhang}
\ead{zhanglei@ncst.edu.cn}	
\author[focal]{Tao Yang\corref{cor1}}
\ead{taoyang@hust.edu.cn}	

\address[rvt]{River Valley Technologies, SJP Building,Cotton Hills, Trivandrum, Kerala, India 695014}
\address[focal]{River Valley Technologies, 9, Browns Court,Kennford, Exeter, United Kingdom}
\cortext[cor1]{Corresponding authors}

\begin{abstract}
%% Text of abstract
Herein, an optimization system of a 1000\,MW double-reheat ultra-supercritical (USC) unit with turbine–extraction-heated air preheaters (EAPHs), which use turbine extractions as heat sources of air preheaters while economizer outlet flue gas’ heat absorbed by low-pressure economizers (LPEs), is designed based on a 1000-MW double-reheat USC unit (reference system). The energetic performance analyses of the USC system with EAPH and reference system are used to compare their major parameters. In addition, thermodynamic analyses under partial load operation conditions are presented . The results show that the proposed system reduced the exergy loss in the air-heating and flue gas-cooling processes. Furthermore, the optimization system increases the temperature of secondary air and reduces the overall superheat degree of several extractions. Theoretically, the proposed system reduces the standard coal equivalent (SCE) consumption by nearly 5.5\,g/kWh under THA load when the temperature of the flue gas entering the electrostatic precipitator is set to 95$^\circ$C. Moreover, the proposed system can still reduce SCE consumption by 2.9\,g/kWh under the condition of 50\% THA load. Our findings indicate that system with APOB could improves the performance of the unit and may provide a theoretical basis for the optimization of double-reheat USC units.
\end{abstract}

\begin{keyword}
%% keywords here, in the form: keyword \sep keyword

%% PACS codes here, in the form: \PACS code \sep code

%% MSC codes here, in the form: \MSC code \sep code
%% or \MSC[2008] code \sep code (2000 is the default)
Double reheat \sep Ultra-supercritical power plant \sep Exergy analysis \sep System Optimization \sep Thermodynamic analyses
\end{keyword}

\end{frontmatter}

%% \linenumbers

%% main text
\section{section1}
dsadsadsadadsas\cite{Brooks1983CHARMM}
\label{}

%% The Appendices part is started with the command \appendix;
%% appendix sections are then done as normal sections
%% \appendix

%% \section{}
%% \label{}

%% If you have bibdatabase file and want bibtex to generate the
%% bibitems, please use
%%
%%  \bibliographystyle{elsarticle-harv} 
%%  \bibliography{<your bibdatabase>}

%% else use the following coding to input the bibitems directly in the
%% TeX file.

%\begin{thebibliography}{00}

%% \bibitem[Author(year)]{label}
%% Text of bibliographic item

%
\section{References}
\section*{Acknowledgement}
This is the 
\clearpage
\section*{References}
\bibliographystyle{elsarticle-num}
\bibliography{1}
%\end{thebibliography}
\end{document}

\endinput
%%
%% End of file `elsarticle-template-harv.tex'.
